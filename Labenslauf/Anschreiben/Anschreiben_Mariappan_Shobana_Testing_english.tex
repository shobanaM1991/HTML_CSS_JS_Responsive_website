%% start of file `template.tex'.
%% Copyright 2006-2013 Xavier Danaux (xdanaux@gmail.com).
%
% This work may be distributed and/or modified under the
% conditions of the LaTeX Project Public License version 1.3c,
% available at http://www.latex-project.org/lppl/.


\documentclass[11.5pt,a4paper,sans]{moderncv}        % possible options include font size ('10pt', '11pt' and '12pt'), paper size ('a4paper', 'letterpaper', 'a5paper', 'legalpaper', 'executivepaper' and 'landscape') and font family ('sans' and 'roman')

% moderncv themes
\moderncvstyle{classic}                            % style options are 'casual' (default), 'classic', 'oldstyle' and 'banking'
\moderncvcolor{blue}                              % color options 'blue' (default), 'orange', 'green', 'red', 'purple', 'grey' and 'black'
%\renewcommand{\familydefault}{\sfdefault}         % to set the default font; use '\sfdefault' for the default sans serif font, '\rmdefault' for the default roman one, or any tex font name
\nopagenumbers{}     
\def\Plus{\texttt{+}}
\def\Minus{\texttt{-}}                             % uncomment to suppress automatic page numbering for CVs longer than one page

% character encoding
\usepackage[utf8]{inputenc}                       % if you are not using xelatex ou lualatex, replace by the encoding you are using
%\usepackage{CJKutf8}   
%\usepackage{dirtytalk}
% if you need to use CJK to typeset your resume in Chinese, Japanese or Korean
\usepackage[english,ngerman]{babel}
\usepackage{datetime} %german format
% adjust the page margins
\usepackage[scale=0.80]{geometry}
%\usepackage[scale=0.85, left=1.2cm, right=1.2cm]{geometry} % Reduce document margins
%\setlength{\hintscolumnwidth}{3cm}                % if you want to change the width of the column with the dates
%\setlength{\makecvtitlenamewidth}{10cm}           % for the 'classic' style, if you want to force the width allocated to your name and avoid line breaks. be careful though, the length is normally calculated to avoid any overlap with your personal info; use this at your own typographical risks...
\usepackage{graphicx} %for signature

\usepackage[document]{ragged2e}

% personal data
\name{\small Shobana}{Mariappan}
\title{Cover Letter}                               % optional, remove / comment the line if not wanted
\address{Wofskehle. 26,}{95326 Kulmbach}%{Deutschland}% optional, remove / comment the line if not wanted; the "postcode city" and and "country" arguments can be omitted or provided empty
\phone[mobile]{+49~152~2996~2032}                   % optional, remove / comment the line if not wanted
%\phone[fixed]{+2~(345)~678~901}                    % optional, remove / comment the line if not wanted
%\phone[fax]{+3~(456)~789~012}                      % optional, remove / comment the line if not wanted
\email{shobanamariappan1991@gmail.com}                               % optional, remove / comment the line if not wanted
%\homepage{www.johndoe.com}                         % optional, remove / comment the line if not wanted
%\extrainfo{additional information}                 % optional, remove / comment the line if not wanted
%\photo[64pt][0.4pt]{picture}                       % optional, remove / comment the line if not wanted; '64pt' is the height the picture must be resized to, 0.4pt is the thickness of the frame around it (put it to 0pt for no frame) and 'picture' is the name of the picture file
%\quote{Some quote}                                 % optional, remove / comment the line if not wanted

% to show numerical labels in the bibliography (default is to show no labels); only useful if you make citations in your resume
%\makeatletter
%\renewcommand*{\bibliographyitemlabel}{\@biblabel{\arabic{enumiv}}}
%\makeatother
%\renewcommand*{\bibliographyitemlabel}{[\arabic{enumiv}]}% CONSIDER REPLACING THE ABOVE BY THIS

% bibliography with mutiple entries
%\usepackage{multibib}
%\newcites{book,misc}{{Books},{Others}}
%----------------------------------------------------------------------------------
%            content
%----------------------------------------------------------------------------------
\begin{document}
%-----       letter       ---------------------------------------------------------
% recipient data
\recipient{\small Beliani GmbH}{Forchheim \\ Deutschland}
\subject{\small Bewerbung als Softwareingenieur Testautomatisierung}
\date{\small \today}

\opening{\vspace{-1mm} \small Sehr geehrte Damen und Herren,} % Sehr geehrte Frau Madlen Schreiner,}  
\closing{\vspace{1mm} \small Mit freundlichen Grüßen \\ %\includegraphics[width=4cm]{Sign.pdf}
\vspace{-14mm}}
%\enclosure[\vspace{-7mm} \small Attachment]{\small Resume, Testimonies}         % use an optional argument to use a string other than "Enclosure", or redefine \enclname
\makelettertitle
\vspace*{-8mm}
\justify
I saw your job advertisement in the online portal with great interest, as my field of interest lies in the area of development and testing. As a Master's graduate in Computational Engineering with practical knowledge in programming and testing, I find that the job described fits very well with my interests and skills.
        
Through my work at Glen Dimplex, I have very good experience with Python programming. I was involved in a project called "Qualifying the right pressure sensors using sensor test data"\, where I tested and analysed the performance of pressure sensors from different manufacturers. To prepare the test data for analysis, I used Python with data processing libraries e.g. Pandas and Numpy to clean the data and visualise it with Matplotlib and Seaborn. At the end of the project completion, I used TKINTER and PyQt5 to create a small Windows tool that receives the sensor data information in CSV format to process it in the background and create the interactive plots for business decisions. For my programming tasks, I used the open-source IDE "Visual Studio Code"\ from Microsoft, which offers a variety of plugins for programming in different languages. 

In my bachelor studies I was interested in testing software. In a short training I got to know different methods of software testing as theory. After the apprenticeship I worked as a software engineer at Accenture. I got good experience in manual testing through client projects at Bank of America, where I developed test cases for online banking applications based on client requirements. At the start of testing, I reviewed requirements and design documents, gathered information from clients and prepared test data. During test execution, I identified and logged the bugs, tracked them until they were fixed, and closed the bugs by retesting. After test execution, I identified tests for the automation tests and wrote scripts for the automation tests using the Selenium Unittest framework. In addition to the tests, I created a status report with the status of the test cases and the errors that blocked the test cases and shared it with the test manager.

I have basic experience with Jenkins and Docker through a self-chosen project in Selenium with Python, where I created an automation test script for a demo online portal called "Orange HRM Open Source Management"\. Besides the project, I am currently preparing for the ISTQB certification and will receive my certificate soon. 

Furthermore, due to my previous professional activities, I bring along communication and teamwork skills, an independent and structured way of working as well as a high degree of self-initiative. Very good knowledge of English and good German round off my profile. With my expertise and skills, I would like to contribute to your innovative company and thus to the success of XXX and its products.

I am available for personal interviews and look forward to convincing you personally of my strengths.



\makeletterclosing
\vspace{0.5mm}

\end{document}
%mit großem interesse habe ich ihr stellenangebot auf ihrer Website gelesen, da mein Interessengebiet die Entwicklung und Test sind. Ich habe meinen Master in Computational Engineering von der Friedrich-Alexander-Universität, Erlangen-Nürnberg, Deutschland. Ich habe die Fähigkeit, selbständig zu arbeiten, Probleme rechtzeitig und genau zu lösen. Aus diesem Grund bewerbe ich mich bei Ihnen als Software Test Ingenieur in Festanstellung.

%Ich arbeitete ein Jahr lang bei der GlenDimplex Deutschland GmbH als Laboringenieur. Dort habe Ich in einem Projekt mit dem Titel "Qualifizierung der richtigen Drucksensoren anhand von Sensortestdaten" gearbeitet, in dem ich die Leistung von Drucksensoren verschiedener Hersteller getestet und analysiert habe. Um die Testdaten für die Analyse zu verarbeiten, habe ich Python-Programmierung mit Datenverarbeitungsbibliotheken wie Pandas und Numpy verwendet, um die Daten zu bereinigen und sie mit Matplotlib und Seaborn zu visualisieren.

%Während des Projektabschlusses wurde ich gebeten, ein kleines Windows-Tool mit TKINTER und PyQt5 zu erstellen, das die Sensordateninformationen im CSV-Format erhält, um sie im Hintergrund zu verarbeiten und die interaktiven Plots für Geschäftsentscheidungen zu erstellen. Für meine Programmieraufgaben verwendete ich die Open-Source IDE "Visual Studio Code" von Microsoft, die eine Vielzahl von Plugins für die Programmierung in verschiedenen Sprachen bietet.

%Nach meinem Bachelor-Abschluss habe ich drei Jahre lang als Software-Ingenieur bei Accenture Services Limited, Indien, gearbeitet. Ich habe an den detaillierten Testprozessen gearbeitet, die ein Projekt von der Anfangsphase bis zur endgültigen Freigabe begleiten. Dies beinhaltet das Verstehen von Business Requirements Documents (BRD) und Functional Requirements Documents (FRD), Testplanung, Testfallentwicklung, Testdurchführung und Fehlerprotokollierung. In Gesprächen mit Kunden habe ich gute Erfahrungen darin gesammelt, die Anforderungen des Kunden zu verstehen und  Use cases für Tests zu erstellen. Für die Automatisierungstests habe ich das Selenium-Python-Framework Unittest verwendet, um Automatisierungstestfälle mit Konzepten zur Page Object Modelling zu entwickeln. Ich habe das Tool JIRA verwendet, um die Testfälle zu schreiben, die Tests auszuführen und Fehler zu protokollieren. Ich habe QC verwendet, um manuelle Tests, Fehlerverfolgung und Fehlerprotokollierung in den Bereichen Mobile Testing, Browser Testing und Debit, Credit Card Testing durchzuführen.

%Ich arbeite derzeit an einem Freien Projekt, schreibe Testfälle und führe die Testfälle anschließend in der Jenkins-Pipeline aus. Neben der Programmierung bereite ich mich auf die ISTQB-Zertifizierung vor. Ich habe Grundkenntnisse in Docker und gute Kenntnisse in Git. Ich möchte Ideen zu Prototypen austauschen. Ich schaffe es, selbständig mit einem planmäßigen Ansatz zu arbeiten, und ich glaube, dass der Austausch von Ideen mit Ihrem Team bessere Ergebnisse bringt. Wenn Sie mehr über mich oder meine Berufserfahrung erfahren möchten, können Sie mich gerne kontaktieren.
